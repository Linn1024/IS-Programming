\documentclass[10pt]{article}

\usepackage[T2A]{fontenc} 
\usepackage[utf8]{inputenc} 
\usepackage[english, russian]{babel} 
\usepackage[a4paper,left=2cm,right=2cm,top=2.5cm,bottom=2.5cm]{geometry}
\usepackage{listings}
\usepackage{caption}
\usepackage{color}
\usepackage{enumerate}
\usepackage{xcolor}
\usepackage{amsbsy}

\title{Требования к оформлению кода на C++}
\author{}
\date{}

\definecolor{string}{HTML}{61a1b0} % цвет строк в коде
\definecolor{comment}{HTML}{61a1b0} % цвет комментариев в коде
\definecolor{keyword}{HTML}{007021} % цвет ключевых слов в коде
\definecolor{morecomment}{HTML}{8000FF} % цвет include и других элементов в коде
\definecolor{сaptiontext}{HTML}{FFFFFF} % цвет текста заголовка в коде
\definecolor{сaptionbk}{HTML}{999999} % цвет фона заголовка в коде
\definecolor{bk}{HTML}{FFFFFF} % цвет фона в коде
\definecolor{frame}{HTML}{999999} % цвет рамки в коде
\definecolor{brackets}{HTML}{B40000} % цвет скобок в коде
\definecolor{digits}{HTML}{40A070} % цвет скобок в коде


% Настройки отображения кода
 
\lstset{
	language=C++, % Язык кода по умолчанию
	% Цвета
	keywordstyle=\color{keyword}\ttfamily\bfseries,
	stringstyle=\color{string}\ttfamily,
	commentstyle=\color{comment}\ttfamily\itshape,
	morecomment=[l][\color{morecomment}]{\#}, 
	% Настройки отображения     
	breaklines=true, % Перенос длинных строк
	basicstyle=\ttfamily\footnotesize, % Шрифт для отображения кода
	tabsize=3, % Размер табуляции в пробелах
	xleftmargin=1cm,
	% Для отображения русского языка
	extendedchars=true,
	literate={Ö}{{\"O}}1
	{Ä}{{\"A}}1
	{Ü}{{\"U}}1
	{ß}{{\ss}}1
	{ü}{{\"u}}1
	{ä}{{\"a}}1
	{ö}{{\"o}}1
	{~}{{\textasciitilde}}1
	{а}{{\selectfont\char224}}1
	{б}{{\selectfont\char225}}1
	{в}{{\selectfont\char226}}1
	{г}{{\selectfont\char227}}1
	{д}{{\selectfont\char228}}1
	{е}{{\selectfont\char229}}1
	{ё}{{\"e}}1
	{ж}{{\selectfont\char230}}1
	{з}{{\selectfont\char231}}1
	{и}{{\selectfont\char232}}1
	{й}{{\selectfont\char233}}1
	{к}{{\selectfont\char234}}1
	{л}{{\selectfont\char235}}1
	{м}{{\selectfont\char236}}1
	{н}{{\selectfont\char237}}1
	{о}{{\selectfont\char238}}1
	{п}{{\selectfont\char239}}1
	{р}{{\selectfont\char240}}1
	{с}{{\selectfont\char241}}1
	{т}{{\selectfont\char242}}1
	{у}{{\selectfont\char243}}1
	{ф}{{\selectfont\char244}}1
	{х}{{\selectfont\char245}}1
	{ц}{{\selectfont\char246}}1
	{ч}{{\selectfont\char247}}1
	{ш}{{\selectfont\char248}}1
	{щ}{{\selectfont\char249}}1
	{ъ}{{\selectfont\char250}}1
	{ы}{{\selectfont\char251}}1
	{ь}{{\selectfont\char252}}1
	{э}{{\selectfont\char253}}1
	{ю}{{\selectfont\char254}}1
	{я}{{\selectfont\char255}}1
	{А}{{\selectfont\char192}}1
	{Б}{{\selectfont\char193}}1
	{В}{{\selectfont\char194}}1
	{Г}{{\selectfont\char195}}1
	{Д}{{\selectfont\char196}}1
	{Е}{{\selectfont\char197}}1
	{Ё}{{\"E}}1
	{Ж}{{\selectfont\char198}}1
	{З}{{\selectfont\char199}}1
	{И}{{\selectfont\char200}}1
	{Й}{{\selectfont\char201}}1
	{К}{{\selectfont\char202}}1
	{Л}{{\selectfont\char203}}1
	{М}{{\selectfont\char204}}1
	{Н}{{\selectfont\char205}}1
	{О}{{\selectfont\char206}}1
	{П}{{\selectfont\char207}}1
	{Р}{{\selectfont\char208}}1
	{С}{{\selectfont\char209}}1
	{Т}{{\selectfont\char210}}1
	{У}{{\selectfont\char211}}1
	{Ф}{{\selectfont\char212}}1
	{Х}{{\selectfont\char213}}1
	{Ц}{{\selectfont\char214}}1
	{Ч}{{\selectfont\char215}}1
	{Ш}{{\selectfont\char216}}1
	{Щ}{{\selectfont\char217}}1
	{Ъ}{{\selectfont\char218}}1
	{Ы}{{\selectfont\char219}}1
	{Ь}{{\selectfont\char220}}1
	{Э}{{\selectfont\char221}}1
	{Ю}{{\selectfont\char222}}1
	{Я}{{\selectfont\char223}}1
	{і}{{\selectfont\char105}}1
	{ї}{{\selectfont\char168}}1
	{є}{{\selectfont\char185}}1
	{ґ}{{\selectfont\char160}}1
	{І}{{\selectfont\char73}}1
	{Ї}{{\selectfont\char136}}1
	{Є}{{\selectfont\char153}}1
	{Ґ}{{\selectfont\char128}}1 
	{0}{{{\color{digits}0}}}1
	{1}{{{\color{digits}1}}}1
        {2}{{{\color{digits}2}}}1
        {3}{{{\color{digits}3}}}1
        {4}{{{\color{digits}4}}}1
        {5}{{{\color{digits}5}}}1
        {6}{{{\color{digits}6}}}1
        {7}{{{\color{digits}7}}}1
        {8}{{{\color{digits}8}}}1
        {9}{{{\color{digits}9}}}1
	{\{}{{{\color{brackets}\{}}}1 % Цвет скобок {
	{\}}{{{\color{brackets}\}}}}1 % Цвет скобок }
}

\begin{document}
\maketitle

\section{Форматирование}

\begin{enumerate}

\item Используйте два или четыре пробела для отступов или символ табуляции для отступов. Отступы внутри одной программы должны быть одинакового размера.

\item Функции разделяйте одной пустой строкой. Смысловые блоки внутри одной функции разделяйте пустой строкой.

\item Если вы используете константу более одного раза, объявите ее, как таковую.

\item Cкобки не отделяются пробелами с внутренней стороны. Между функцией и ее аргументами пробел не ставится.

\begin{lstlisting}
spam(ham[1]); // Правильно
spam( ham [ 1 ] ); // Неправильно
spam (ham[1]); // Неправильно
\end{lstlisting}

\item Перед запятой пробел не ставится, после — ставится.

\begin{lstlisting}
printf("%d %d", x, y); // Правильно
printf("%d %d" , x , y); // Неправильно
printf("%d %d",x,y); // Неправильно
\end{lstlisting}

\item Всегда окружайте следующие бинарные операторы ровно одним символом пробела с каждой стороны:

\begin{enumerate}[1.]
	\item присваивания (\texttt{=, +=, -=} и т. д.)
	\item сравнения (\texttt{==, <, >, !=, <=, >=})
	\item логические (\texttt{\&, |, \&\&, ||})
	\item арифметические (\texttt{+, -, *, /, \%})
\end{enumerate}

\item Не располагайте несколько инструкций в одной строке. Разнесите их по разным строкам.

\begin{lstlisting}
x = 35;
func(10); // Правильно
x = 35, func(10); // Неправильно
\end{lstlisting}

\item Не располагайте блок из нескольких инструкций на одной строке сразу после if, while и т. д.

\item Ширина строки не должна превышать 80-120 символов.

\item Объявление переменных должно быть максимально близко к месту их использования.

\item Открывающие фигурные скобки можно переносить на следующую строку, можно не переносить, но нужно придерживаться одного стиля. Если не переносите, то ставьте пробел.

\item Нужно ставить пробел перед скобками и после скобок стандартных операторов (if, while и т. д.)

\begin{lstlisting}
while (true) { // Правильно
while(true){ // Неправильно
\end{lstlisting}


\end{enumerate}

\section{Имена}

\begin{enumerate}

\item Не рекомендуется использовать символы «l», «O», и «I» как имена переменных. В некоторых шрифтах они могут быть очень похожи на цифры.

\item Название переменной должно содержать информацию о ее назначении, не допускается использование названий из одной буквы, за исключением счетчиков в циклах (i, j, k) и переменных, хранящих размер входных данных (n, m).

\item Давайте переменным говорящие английские имена, не используйте транслит.

\begin{lstlisting}
cin >> num_letters; // Правильно
cin >> kolvo_bukv; // Неправильно
\end{lstlisting}

\item Имена констант должны содержать только заглавные буквы. Названия классов и структур должны начинаться с заглавной буквы. Названия
переменных должны начинаться с маленькой буквы. Функции и методы начинаются с маленькой буквы. Слова разделяются символами подчёркивания или с помощью CamelCase. Необходимо придерживаться одного
стиля в программе.
Примеры: \texttt{const int NAME, const int NAME\_WITH\_SEVERAL\_WORDS\_IN\_IT, struct Struct, class Class\_with\_several\_words, int var, int varWithSeveralWords}.

\end{enumerate}

\end{document}
